\documentclass[10pt,twocolumn,letterpaper]{article}

\usepackage{cvpr}
\usepackage{times}
\usepackage{epsfig}
\usepackage{graphicx}
\usepackage{amsmath}
\usepackage{amssymb}

\newcommand{\tk}[1]{\textcolor{red}{TK: #1}}
\newcommand{\ao}[1]{\textcolor{green}{AO: #1}}
\newcommand{\tsj}[1]{\textcolor{magenta}{TSJ: #1}}
\newcommand{\vm}[1]{\textcolor{blue}{VM: #1}}

% Include other packages here, before hyperref.

% If you comment hyperref and then uncomment it, you should delete
% egpaper.aux before re-running latex.  (Or just hit 'q' on the first latex
% run, let it finish, and you should be clear).
\usepackage[pagebackref=true,breaklinks=true,colorlinks,bookmarks=false]{hyperref}

\def\cvprPaperID{7813}
\def\httilde{\mbox{\tt\raisebox{-.5ex}{\symbol{126}}}}

\begin{document}

\title{Transfer Learning in Visual and Relational Reasoning -- Rebuttal}
\maketitle
\thispagestyle{empty}

\section{Reviewer \#2 questions}

\begin{enumerate}
  \item What are $vo_tt$ and $mo_t$ and how they differ (lines 400-403)? What is the difference between both objects? \vm{$vo_t$ is the result of a retrieval operation on the content of the current input frame. $mo_t$ is the result of a similar operation on the memory content. The difference between both objects is thus their source.}

  \item How the temporal classifier $\tau_t$ is trained and used? Do you use its logits or classes directly (lines 389-397)?
  \item What are $va_t$ and in general all other symbols. How they are computed. \vm{$va_t$ is the visual attention vector, as defined on lines 412 - 413. $va_t$ is the result of a softmax layer}.
  \item What is pseudo-attention (line 453)? Why such a name? \vm{$w_t$ is called a pseudo-attention vector as its norm may not always be 1 (just like a probability vector), but rather vary between 0 (no change) to 1 (complete change)}.
  \item What are reasoning operations (lines 389-391)? \vm{One iteration of the SAM Cell is considered to be a reasoning operation, as defined on lines 376 - 377}.
  \item How many 'reasoning operations' (what is k)? \vm{8 reasoning steps, as noted on lines 532-533.}
  \item Figure 4 shows results on CLEVR-CoGenT. CLEVR-CoGenT uses a transfer between objects of type A and objects of type B (e.g. different combination of shapes and colors). Is this used here? Or this is standard CLEVR results? What are the results on the regular CLEVR, is it 95\% from the supp. material? \vm{The results presented in Fig 4 are reported on the condition A of CLEVR-CoGenT. No results are reported for CLEVR.}
  \item Why the paper doesn't compare to other methods on CLEVR?
\end{enumerate}



% References
% {\small
% \bibliographystyle{ieee}
% \bibliography{}
% }

\end{document}

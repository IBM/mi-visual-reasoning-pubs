\documentclass{article}
\usepackage{neurips_2019}

\usepackage[utf8]{inputenc} % allow utf-8 input
\usepackage[T1]{fontenc}    % use 8-bit T1 fonts
\usepackage[usenames, svgnames]{xcolor}
\usepackage{hyperref}       % hyperlinks
\usepackage{bchart}
\usepackage{pgfplots}
\pgfplotsset{width=7 cm,compat=1.5}
\hypersetup{
	colorlinks=true,
	linkcolor={red!80!black},
	%	citecolor={blue!80!black},
	%	urlcolor={magenta}
}

\usepackage{url}            % simple URL typesetting
\usepackage{booktabs}       % professional-quality tables
\usepackage{amsfonts}       % blackboard math symbols
\usepackage{nicefrac}       % compact symbols for 1/2, etc.
\usepackage{microtype}      % microtypography
\usepackage{graphicx}

% Useful packages
\usepackage{amsmath, amssymb, amsfonts, bm}
\usepackage{amsthm}
\usepackage{mathtools}
%\usepackage[usenames, dvipsnames]{color}
\usepackage{cleveref}
\crefname{figure}{Figure}{Figures}

%\usepackage{enumitem}
%\setlist[enumerate]{labelindent=5pt, label=\alph*)} 


% \usepackage{blindtext}

\newtheorem{theorem}{Theorem}
\newtheorem{proposition}[theorem]{Proposition}
\newtheorem{corollary}[theorem]{Corollary}
\newtheorem{lemma}[theorem]{Lemma}
\newtheorem{remark}[theorem]{Remark}

\theoremstyle{remark}
\newtheorem*{notation}{Notation}
%\newtheorem*{remark}{Remark}
\newtheorem*{note}{Note}

\theoremstyle{definition}
\newtheorem{definition}[theorem]{Definition}
\newtheorem{fact}{Fact}
\newtheorem*{observation}{Observation}
\newtheorem*{condition}{Condition}
\newtheorem*{claim}{Claim}
\newtheorem*{example}{Example}
\newtheorem*{question}{Question}


\newcommand{\colorbx}[1]{\medskip\noindent\scalebox{1.05}{\fcolorbox{SaddleBrown}{white}{\color{SteelBlue}{\textbf{#1}}}}}

\newcommand{\tk}[1]{\textcolor{red}{TK: #1}}
\newcommand{\ao}[1]{\textcolor{green}{AO: #1}}
\newcommand{\tsj}[1]{\textcolor{magenta}{TSJ: #1}}
\newcommand{\va}[1]{\textcolor{blue}{Vincent A: #1}}

\newcommand{\Reals}{\mathbb{R}}
\newcommand{\T}{\mathsf{T}}
\DeclareMathOperator{\softmax}{\mathrm{softmax}}

\newcommand{\cA}{\mathcal{A}}
\newcommand{\cG}{\mathcal{G}}
\newcommand{\cK}{\mathcal{K}}
\newcommand{\cH}{\mathcal{H}}
\newcommand{\cI}{\mathcal{I}}

\newcommand{\vect}[1]{\mathbf{#1}}
\newcommand{\vx}{\vect{x}}
\newcommand{\vy}{\vect{y}}
\newcommand{\vone}{\vect{1}}
\newcommand{\proj}{\mathrm{proj}}

\newcommand{\imatch}{g^{\mathrm{vis}}}
\newcommand{\mmatch}{g^{\mathrm{mem}}}

\newcommand{\doadd}{h^{\mathrm{add}}}
\newcommand{\doreplace}{h^{\mathrm{repl}}}

\newcommand{\tlast}{\tau^{\mathrm{last}}}
\newcommand{\tlatest}{\tau^{\mathrm{latest}}}
\newcommand{\tnow}{\tau^{\mathrm{now}}}
\newcommand{\tnone}{\tau^{\mathrm{none}}}

\newcommand{\rhead}{w^{\mathrm{read}}}
\newcommand{\whead}{w^{\mathrm{write}}}


\title{Visually Grounded Reasoning about Temporal Concepts in Selective Attention Memory}

\author{T.S. Jayram, Vincent Albouy, Tomasz Kornuta, Emre Sevgen, Ahmet Ozcan}

\begin{document}
	
	\maketitle
	\begin{abstract}
	Visual reasoning in videos requires understanding temporal concepts in addition to the objects and their relations in a given frame.  Selective attention and memory are the essential faculties, which humans rely on to accomplish this task.  Here we present the Selective Attention Memory network (SAM Net), which is an end-to-end differentiable recurrent model equipped with external memory.  SAMNet can perform multi-step reasoning on a frame-by-frame basis, and dynamically control information flow to the memory to store context-relevant representations to answer questions. We tested our model on the COG dataset (a multi-frame visual question answering test), and outperformed the state of the art baseline for hard tasks and in terms of generalization.
%		We introduce the Selective Attention Memory Network (SAMNet), a end-to-end differentiable architecture for video reasoning. It is a recurrent model with an external memory that enables frame by frame reasoning over text and video. 
%		We show SAMNet's abilities on the COG dataset made for Video Question Answering (Guangyu Robert Yang, Igor Ganichev et al., ECCV 2018). We compare our model to the original COG model and show that SAMNet outperforms the COG model especially on the hardest version of the dataset with longer sequences and a maximum number of distractors. We also demonstrate that our model has good generalization capabilities going from easy to hard tasks.
		
		
		
		
	\end{abstract}
	
	
	
	
	
	
	
	
	\section{Introduction}

In recent years there has been substantial progress in sys-tems  that  can  find  factual  answers  in  text,  starting  withIBM’s Watson system~\cite{ferrucci2010building}

, and now with high-performing neural systems that can answer short ques-tions provided they are given a text that contains the answer e.g.~\cite{wang2018glue}


AI  has  achieved  remarkable  mastery  over  games  such  asChess, Go, and Poker, and evenJeopardy!, but the rich variety of standardized exams has remained a landmark chal-lenge.   Even  in  2016,  the  best  AI  system  achieved  merely 59.3\% on an 8th Grade science exam challenge (Schoenicket al., 2016).




Playing Atari Games~\cite{mnih2015human}


Despite several successes across many domains Deep learning~\cite{lecun2015deep} still struggles with

learning algoritms
learning reasoning~\cite{graves2016hybrid}


Wingrad Scheme challenge~\cite{levesque2012winograd}

visual reasoning~\cite{mogadala2019trends} - datasets such as COG~\cite{yang2018dataset} and 
SVQA (Synthetic Video Question Answering)~\cite{song2018explore}


ARISTO project~\cite{clark2019f} - based on RoBERTa~\cite{liu2019roberta} contextual word embeddings

transformer-based solutions~\cite{vaswani2017attention} using self-attention


“The current neural network approaches will find it difficult to determine which combinations of ‘later’, ‘earlier’, ‘more’, and ‘less’ constitute ‘increase’ and which constitute ‘decrease,'” Davis says. “Neural networks have no inherent idea of magnitude or of time.”
~\cite{davis2016write}

bAbI~\cite{weston2015towards}

Visual Dialog~\cite{das2017visual} - the same, they keep the whole history of the dialog in memory

\begin{itemize}
\item bAbI:  MemNets~\cite{weston2014memory} have access to the whole story at once
\item the same goes to SoftPats~\cite{haurilet2019s} - they build graph per frame and then frame number is treated as one dimensions, so at the end the \textit{Traveler} can access all of them at the same time 
\item The paper~\cite{le2019learning} focuses on SVQA and TGIF-QA -  they access all frames at once, i.e. cut the video into clips, process each frame with CNNs and then aggregate feature representations of equal-size clips obtained by a temporal attention mechanism. So in fact the model has access to all frames all the time.
\end{itemize}
so the time aspect is really... not dealt with?

Additionally, in~\cite{song2018explore} the authors introduced a large-scale dataset caled SVQA (Synthetic Video Question Answering) consisting of (Total QA pairs: 83160/11760/23760 and Total Videos 8400/1200/2400).
As "using all frames is time-consuming. Thus we divide each video into clips (segments) of 16 frames, with 80\% overlap between successive clips (segments)" and "We extract feature from each clip and aggregate features of all clips from one video to form a sequential video representation." -- which means that they identified the problem that you "cannot extract features from all frames" at the beginning and pass that to the model. But instead of proposing a solution that will deal with the video on per-frame basis, they "cheated". ;)

IMPORTANT: \textbf{we do not have any explicit assumptions when it comes to number of frames/length of the movie/number of distractionts}, so there is no need for cutting video into cuts etc.

\subsection{Contributions}

\begin{itemize}
\item \textbf{Time aspect}:
\begin{itemize}
\item Learning the temporal association - grounding the time-related words with meaning
\item Learning the concept of time
\item time context being by-product of gates
\end{itemize}

\item Visual Grounding:
\begin{itemize}
\item Learning complex, multi-step reasoning that involves grounding of words and visual representations/objects
\item 
\end{itemize}
\item \textbf{Selective Attention Memory}:
\begin{itemize}
\item updating the memory content only with relevant visual information depending on the temporal context
\item content based and location based addressing for reading and writing
\item new memory interface/gating designed in such a way enabling the model to control the flow of current visual information and content of the memory in a selective way

\end{itemize}
\end{itemize}








	
	\section{Selective Attention Memory Network (SAMNet)}

%\begin{figure}[!b]
%\begin{minipage}{0.43\textwidth}
%	\centering
%	\includegraphics[width=\textwidth]{../img/architecture/samnet_architecture4}
%\end{minipage}\hfill
%\begin{minipage}{0.55\textwidth}
%	\centering
%	\includegraphics[width=\textwidth]{../img/architecture/samcell_reasoning}
%\end{minipage}\hfill
%\caption{General architecture of SAMNet (left) and a single reasoning step in SAMCell (right)}
%\label{fig:samnet}
%\end{figure}	

%Selective Attention Memory Network (SAMNet) is a end-to-end differentiable model made for video reasoning. It is a model based on attention mechanisms but also on a Selective Attention Memory which is able to store selected entities. This memory enables SAMNet to reason across multiple frames and perform spatio-temporal reasoning. 
%The core of SAMNet is based a recurrent cell called SAMCell. By aligning together a series of k SAMCells per frame, the network can perform k reasoning steps over a frame. At every new frame, a new series of k SAMCells is initiated. The SAMCell can read and write to memory at every frame using a content addressable mechanism. This section describes the model and the different units that composed a SAMCell. They are called the Question-driven Controller, the Visual Retrieval Unit, the Memory Retrieval Unit, the Reasoning unit, the Memory update unit, and the Summary Object Udpate Unit. 
%The model is also composed of an Image Encoder and a Question Encoder both responsible to pre-process the visual and textual inputs. The output unit is a classifier.
%All those modules are described below.

\begin{figure}[hbtp]
	\centering
	\includegraphics[width=\textwidth]{../img/architecture/samnet_architecture4}
	\caption{General architecture of SAMNet}
	\label{fig:samnet}
\end{figure}	

SAMNet is an end-to-end differentiable recurrent model equipped with an external memory for enabling multi-step reasoning over text and video (\cref{fig:samnet}).
The memory is used to store relevant objects representing contextual information about words in text and visual objects in video frames. 
Each address of the memory stores a $d$-dimensional vector, where $d$ is a global parameter.
The memory  can be accessed through either content-addressing, via dot-product attention, or location-based addressing. 
Coupled with gating mechanisms to be described later, this enables correct objects to be retrieved 
in order to perform spatio-temporal reasoning on frames and text. 
%A notable feature of this design is that the number of addresses $N$ can be set to different values during training and 
%testing to fit the characteristics of data.

\begin{figure}[hbtp]
	\centering
	\includegraphics[width=\textwidth]{../img/architecture/samcell_reasoning}
	\caption{Single reasoning step in SAMCell}
	\label{fig:samcell}
\end{figure}	

The core of SAMNet is a recurrent cell called SAMCell (\cref{fig:samcell}). 
By aligning together a new series of $k$ SAMCells per frame, the network can perform $k$ 
reasoning steps over each frame, with information flowing between frames through the external memory. 
While processing a single frame, for $t=1,2, \dots, k$, SAMCell maintains the following information as part of its recurrent state:
(a) $c_t \in \Reals^d$, the control state used to drive the reasoning over objects in the frame and memory; and
(b) $so_t  \in \Reals^d$, the summary visual object representing the relevant object for step $t$.
Let $M_t \in  \Reals^{N \times d}$ denote the external memory with $N$ slots at the end of step $t$.
Let $\whead_t \in  \Reals^N$ denote an attention vector over the memory locations;
in a trained model, $\whead_t$ points to the location of first empty slot in memory for adding new objects.   


This section describes the model and the different units that composed a SAMCell. They are called the Question-driven Controller, the Visual Retrieval Unit, the Memory Retrieval Unit, the Reasoning unit, the Memory update unit, and the Summary Object Udpate Unit. 
The model is also composed of an Image Encoder and a Question Encoder both responsible to pre-process the visual and textual inputs. The output unit is a classifier.
All those modules are described below.














The  Summary Unit is the last unit of the SAMCell. It is responsible to output the new summary object. It first picks which object is relevant between the object extracted from memory and the visual object extracted from the image. Once the relevant object is picked, it is combined with the former summary object through a linear layer to become the new summary object. It is the final step of the SAMCell reasoning cycle. 

The image encoder, question encoder and output unit are described in the appendix.

	
	\section{Experiments}

\subsection{Generalization capabilities}
	
	
	\newpage
	\bibliographystyle{alpha}
	\bibliography{../cog_bibliography}
	
	\newpage
	\section{Appendix}



\section{VWM model}

\begin{notation}
Treat 1D tensors as column vectors and 2D tensors as matrices, where appropriate.
We use lower case to represent both 1D and 2D tensors but occasionally use upper case
for 2D tensors where matrix operations are involved.

\begin{enumerate}
	\item Let 
	$\Delta^d = \{ (x_0, x_1, \dots, x_d) : x_0 + x_1 + \dots + x_d = 1, x_i \ge 0, i = 0, 1, \dots, d\}$ denote the standard $d$-simplex.	
	
	\item  Let $\circ$ denote concatenation of two tensors with identical shape except possibly
	for their last dimensions $d_1$ and $d_2$, respectively,  
	resulting in a tensor with last dimension of $d_1+d_2$. 
	
	\item Let $\odot$ denote  element-wise product of two tensors of same shape,
	i.e., Hadamard product for vectors/matrices.
	
	\item Let $\otimes$ denote tensor product of two tensors, 
	i.e. Kronecker product for vectors/matrices.
\end{enumerate}
\end{notation}	

\section{Basic layers/modules}

\colorbx{Linear (Affine) Layer}

\begin{description}
	\item[Inputs:] A tensor $x$ with last dimension $n$.
	\item[Parameter:] An affine function $\cG: \Reals^n \to \Reals^m$ with 
	weight and bias parameters.
		
	\item[Output:] A tensor $y$ with last dimension $m$, and remaining dimensions
	same as that of $x$, obtained by applying $\cG$ to each 1D slice of $x$
	along the last dimension.
	\end{description}


\colorbx{Attention Module}

\begin{description}
	\item[Inputs:] 
	\begin{enumerate}
		\item[]
		\item Query: $q \in \Reals^d$
		\item Keys: $K \in \Reals^{N \times d}$	
		\item Values: $V \in \Reals^{N \times d}$. By default $V=K$, unless mentioned explicitly.	
	\end{enumerate}

	\item[Parameter:] Weight $w \in \Reals^d$

	\item[Outputs:] 
	\begin{enumerate}
		\item[]
		\item Content vector: $h =  V^{\T} u \in \Reals^d$
		\item Attention vector:  $w = \softmax(K(w \odot q)) \in \Reals^N$
	\end{enumerate}
\end{description}


\colorbx{Interaction Module}

\begin{description}
	\item[Inputs:] 
	\begin{enumerate}
		\item[]
		\item Base object: $b \in \Reals^d$
		\item Feature objects: $f \in \Reals^{M \times d}$	
	\end{enumerate}
	
	\item[Parameters:] 
	\begin{enumerate}
	\item[]
	\item Base object projection linear layer: $\cG: \Reals^d \to \Reals^d$
	\item Feature objects projection linear layer : $\cK: \Reals^{M \times d} \to \Reals^{M \times d}$
	\item Modifier linear layer:  $\cH: \Reals^{M \times 2d} \to \Reals^{M \times d}$	
\end{enumerate}
		
	\item[Output:] Modified feature objects 
	$f' =  \cH( \cK(f) \odot ( \vone \otimes \cG(b))) \in  \Reals^{M \times d}$
\end{description}

\hrulefill

\section{VWM cell}

The VWM recurrent cell is executed for $T$ reasoning steps for every frame in
the temporal order.  Within a single frame, the cell state at the end of each reasoning step 
$t=1,2, \dots, T$ is denoted by $(c_t, M_t, o_t)$, where: 
\begin{enumerate}
	\item $c_t \in \Reals^d$ is the control state;
	\item $M_t \in  \Reals^{N \times d}$ is the visual working memory with $N$ slots;
	\item $w_t \in  \Reals^N$ is the write head; and
	\item $so_t  \in \Reals^d$ is the summary visual object.
\end{enumerate} 
The initial state is such that both $c_0$ and $so_0$ are initialized
to a fixed value at the start of each frame. However $M_0$ is initialized only once at
the start of the first frame and otherwise taken to be the value of $M_T$ at the
end of the previous frame.

%The number of slots $N$ for the VWM $M_t$ is not fixed because the neural network 
%parameters do not depend on it. It can be variable across the different datasets used
%for training, validation and test as well as within each dataset. This, for example, enables a 
%form of transfer learning where we can train on an easy dataset for one value of $N$
%and study its generalization to a hard dataset using a larger value of $N$.

\colorbx{Question-driven Controller}

The Question-driven Controller plays an important role in the reasoning process.
It drives the attention over the question and produces the new control states. Each new control state defines a new reasoning operation. The inputs of this unit are the past control state, the question encoding and the contextual words (see Question Encoding Unit). It uses the dot product attention between the contextual words
and the combination of the past control states and the question encoding.  This attention layer produces the new control state.

This unit also outputs the temporal class weights that will be used in the Reasoning Unit. It gives access to a temporal information for the current words (last, latest, now, none temporal).

\begin{description}
	\item[Inputs:] 
	\begin{enumerate}
		\item[]
		\item Reasoning step $t = 1,2, \dots, T$
		\item Previous control state: $c_{t-1} \in \Reals^d$	
		\item Contextual words: $cw \in \Reals^{L \times d}$
		\item Question encoding: $q \in \Reals^d$
	\end{enumerate}
	
	\item[Parameters:] 
	\begin{enumerate}
		\item[]
		\item Reasoning step-dependent linear layer: $\cG_t: \Reals^d \to \Reals^d$, depending on $s$
		\item Concatenation linear layer: $\cH: \Reals^{2d} \to \Reals^d$
		\item Attention module $\cA$
		\item Temporal classifier:  $\cK: \Reals^d \to \Delta^3$. A two-layer feedforward
		network with ELU activation in the hidden layer of $d$ units.	
		The classes for the temporal context are labeled ``last'', ``latest'', ``now'', as well as 
		a fourth class label ``none`` indicating no temporal context.
		If $\tau \in \Delta^3$ is the output of the classifier, we denote the components by
		$\tlast$, $\tlatest$, $\tnow$ and $\tnone$.
	\end{enumerate}
	
	\item[Outputs:] 
	\begin{enumerate}
		\item[]
		\item Control state $c_t \in \Reals^d$
		\item Control attention $ca_t \in \Reals^L$
		\item Temporal class weights $\tau_t \in \Reals^4$
    \end{enumerate}
	
	\item[Equations:] 
	\begin{enumerate}
		\item[]
		\item Modulation: $y = \cH\bigl([c_{t-1}, \cG_t(q)]\bigr)$
		\item Control state and attention: $c_t,  ca_t= \cA(y, cw)$
		\item Temporal classification: $\tau_t = \cK(c_t)$
\end{enumerate}
\end{description}


\colorbx{Visual Retrieval Unit}

The visual retrievial unit is responsible to extract visual information from the current image given a control state coming from the Question-driven Controller. It is first projecting the past summary object and the feature maps together using the interaction module.
It is then using the attention module as follow. The query are the control states and the keys and are the feature maps coming from the image encoder. The results of this attention is applied on the modified features maps coming from the interaction module.  
This unit outputs the extracted object and the visual attention.

\begin{description}
	\item[Inputs:] 
	\begin{enumerate}
		\item[]
		\item Control state: $c_t \in \Reals^d$	
		\item Previous summary object: $so_{t-1} \in \Reals^d$
		\item Feature map of current frame: $F \in \Reals^{H \times W \times d}$
	\end{enumerate}
	
	\item[Parameters:] 
	\begin{enumerate}
		\item[]
		\item Interaction module $\cI$
		\item Attention module $\cA$
	\end{enumerate}
	
	\item[Outputs:] 
	\begin{enumerate}
		\item[]
		\item Visual object: $vo_t \in  \Reals^d$
		\item Visual attention: $va_t  \in \Reals^{H \times W \times d}$
	\end{enumerate}

	\item[Equations:]
	\begin{enumerate}
		\item[]
		\item Modified feature map: $\hat{F} = \cI(so_{t-1}, F)$
		\item Visual object and attention: $vo_t, va_t = \cA(y, \hat{F}, M_{t-1})$
	\end{enumerate}
\end{description}

\begin{note}
	Appropriate flatten/unflatten operations are performed to match the signature 
	of the modules.
\end{note}


\colorbx{Memory Retrieval Unit}

The role of the memory retrieval unit is to read and extract object from memory).
As the Visual Retrieval Unit, it uses the combination of the two following submodules. The interaction module blends together the extracted object and the content of the memory. The attention module then extract the corresponding object in memory if present. This unit outputs the extracted object and its corresponding location, we call it the "read head".

\begin{description}
	\item[Inputs:] 
	\begin{enumerate}
		\item[]
		\item Control state: $c_t \in \Reals^d$	
		\item Previous summary object: $so_{t-1} \in  \Reals^d$
		\item Previous VWM $M_{t-1} \in \Reals^{N \times d}$
	\end{enumerate}
	
	\item[Parameters:] 
	\begin{enumerate}
		\item[]
		\item Interaction module $\cI$
		\item Attention module $\cA$
	\end{enumerate}
	
	\item[Outputs:] 
	\begin{enumerate}
		\item[]
		\item Memory object: $mo_t \in \Reals^d$
		\item Read head: $\rhead_t \in \Reals^N$
	\end{enumerate}

	\item[Equations:]
	\begin{enumerate}
		\item[]
		\item Modified VWM: $\hat{M}_t = \cI(so_{t-1}, M_{t-1})$
		\item Memory object and attention: $mo_t, \rhead_t = \cA(y, \hat{M}_t, M_{t-1})$
	\end{enumerate}
\end{description}


\colorbx{Reasoning Unit}

%control_state, visual_object, memory_object, temporal_class_weights

\begin{description}
	\item[Inputs:] 
	\begin{enumerate}
		\item[]
		\item Control state: $c_t \in \Reals^d$
		\item Visual object: $vo_t \in \Reals^d$
		\item Memory object: $mo_t \in \Reals^d$
		\item Temporal class weights $\tau \in \Delta^3$
	\end{enumerate}
	
	\item[Parameters:] Validator modules $\cG, \cK: \Reals^{2d} \to \Reals$.
	Both $\cG, \cK$ are two-layer networks of $2d$ hidden units,
	using ELU activation in the hidden layer, and sigmoid in the output layer.
	
	\item[Output:] Predicate gates for the current reasoning step
	\begin{enumerate}
    	\item Object match predicate gates (i) image: $\imatch_t \in [0,1]$ and 
    	(ii) memory: $\mmatch_t \in [0,1]$.
		
		\item Memory update predicate gates (i) add: $\doadd_t \in [0,1]$ and
		(ii) replace: $\doreplace_t \in [0,1]$
    \end{enumerate}

	\item[Equations:]
	\begin{enumerate}
		\item[]
		\item $\imatch_t \in [0,1]$:
		It's true if there is a valid visual object. This assumes that
		the current reasoning step refers to either ```now''  or ``latest''.
		
		\item$\mmatch_t \in [0,1]$:
		It's true if there is a valid memory object. This assumes that
		the current reasoning step refers to either ``last'',  or alternatively ``latest'' 
		but there is no matching visual object.
		
		\item $\doadd_t$:
		
		
		\item $\doreplace_t$: 
\end{enumerate}
\end{description}


\colorbx{Memory Update Unit}

This unit is meant to update the content of the memory. 

Three actions can happen:

\begin{itemize}
	\item There is no object to be added to memory, the memory remains unchanged
	\item There is one object that needs to be added to memory, but a similar object is already in memory at a given location. The new object will replace the old object at this location
	
	\item There is one object that needs to be added to memory, and  it is a new object. It is added at the write head location.
\end{itemize}


This module also updates the position of the write head. If a new object as been added to the current write head position, the right head shifts right to a new empty slot. If the object has been replaced, the write head doesn't move.
% visual_object, visual_working_memory, read_head, write_head, do_replace, do_add_new

\begin{description}
	\item[Inputs:] 
	\begin{enumerate}
		\item[]
		\item Visual object: $vo_t \in \Reals^d$
		\item Memory object: $mo_t \in \Reals^d$
		\item Memory update predicate gates:  $\doadd_t, \doreplace_t \in [0,1]$
		\item Read head: $\rhead_t \in \Reals^N$
		\item Previous VWM $M_{t-1} \in \Reals^{N \times d}$
		\item Previous write head: $\whead_{t-1} \in \Reals^N$
	\end{enumerate}

	\item[Outputs:] 
	\begin{enumerate}
		\item[]
		\item VWM $M_t \in \Reals^{N \times d}$
		\item Read head: $\rhead_t \in \Reals^N$
		\item Write head: $\whead_t \in \Reals^N$
	\end{enumerate}

\end{description}



\colorbx{Summary Object Update Unit}

The  Summary Unit is the last unit of the SAMCell. It is responsible to output the new summary object. It first picks which object is relevant between the object extracted from memory and the visual object extracted from the image. Once the relevant object is picked, it is combined with the former summary object through a linear layer to become the new summary object. It is the final step of the SAMCell reasoning cycle. 

The image encoder, question encoder and output unit are described in the appendix.

\begin{description}
	\item[Inputs:] 
	\begin{enumerate}
		\item[]
		\item Previous summary object: $so_{t-1} \in \Reals^d$
		\item Visual object: $vo_t \in \Reals^d$
		\item Memory object: $mo_t \in \Reals^d$
		\item Object predicate gates: $\imatch_t, \mmatch_t \in [0,1]$
	\end{enumerate}
	
	\item[Parameters:] Concatenation linear layer $\cH$ 
	
	\item[Output:] 
     New summary object:
	             $so_t = \cH\bigl([so_{t-1}, (\imatch_t * vo_t + \mmatch_t * mo_t)]\bigr) \in \Reals^d$ 
	
\end{description}


\noindent\makebox[\linewidth]{\rule{\paperwidth}{1pt}}



\begin{figure}
	\includegraphics[width=\textwidth]{img/model2}
	\label{fig:model}
\end{figure}	

\begin{figure}
	\includegraphics[width=\textwidth]{img/image}
	\label{fig:model}
\end{figure}	

\colorbx{Image Encoder}

\colorbx{Question Encoder}

\colorbx{Output Unit}



\section{Training and Implementation Details}

SAMNet is implemented on IBM's Mi-Prometheus~\cite{kornuta2018accelerating} framework based on Pytorch. 
We trained all our models using NVIDIA’s GeForce GTX TITAN X GPUs. SAMNet was trained using 8 reasoning steps and a hidden state size of 128. The external memory has 128-bit slots for all experiments. We trained our model until convergence but we also have set a training time limit of 80 hours.


\begin{table}[t]
	\tiny
	
	\caption{COG test set accuracies for  SAMNet \& COG models. For the COG section, the results marked as 'paper' comes from the original COG paper ~\cite{yang2018dataset}, whereas the results marked as 'ours' come from our own experiments using the following implementation: https://github.com/google/cog }
	
	\resizebox{\textwidth}{!}{
	\centering
	\begin{tabular}{ccccccccccc}
		\toprule
		Model & & SAMNet & && && COG&& \\
		\cmidrule{2-5} \cmidrule{7-11} 
		&&&&& & paper & ours & ours & paper&\\
		\cmidrule{7-9} \cmidrule{10-11}
		Trained on       & canonical & canonical & canonical & hard &           &  canonical  & canonical  & canonical & hard \\ 
		Fine tuned on  & - & - & hard  & - &           & -   & - & hard & - \\ 
		Tested on        & canonical & hard & hard & hard &            &canonical  & hard & hard & hard  \\ 
		\midrule
		
		Overall accuracy & 98.0 & 91.6 & 96.5  & running &           & 97.6  & 65.9 & running& 80.1 \\ 
		
		\midrule 
		
		
		AndCompareColor	&	93.5		&	82.7	&	89.2	&&		&81.9	&57.1&&	51.4
		\\ 
		AndCompareShape	&	93.2 		&	83.7	&	89.7	&&	&	80.0	&53.1	&&50.7\\ 
		AndSimpleCompareColor	&	99.2	&		85.3	&	97.6	&	&	&99.7&	53.4&&	78.2\\ 
		AndSimpleCompareShape	&	99.2&			85.8	&	97.6	&&	&	100.0	&56.7&&	77.9\\ 
		CompareColor	&	98.1		&	89.3	&	95.9	&&		&99.2&	56.1&&	50.1\\ 
		CompareShape	&	98.0	&		89.7	&	95.9	&&	&99.4	&66.8	&&50.5
		\\ 
		Exist	&	100.0	&		99.7	&	99.8		&&	&	100.0&	63.5&&	99.3\\ 
		ExistColor	&	100.0		&	99.6	&	99.9	&&	&	99.0&	70.9&&	89.8\\ 
		ExistColorOf	&	99.9	&		95.5	&	99.7		& & &	99.7&	51.5&&	73.1\\ 
		ExistColorSpace	&94.1		&	88.8	&	91.0	&& &	98.9	&72.8	&&89.2\\ 
		ExistLastColorSameShape	&	99.5		&	99.4	&99.4	&&		&100.0	&65.0&&	50.4
		\\ 
		ExistLastObjectSameObject	&	97.3	&		97.5	&	97.7	&&	&	98.0&	77.5	&&60.2\\ 
		ExistLastShapeSameColor	&	98.2		&	98.5&	98.8	&&	&	100.0&	87.8&&	50.3\\ 
		ExistShape	&	100.0	&	99.5	&	100.0	&&&	100.0&	77.1	&&92.5\\ 
		ExistShapeOf	&	99.4		&	95.9	&	99.2	&&&100.0	&52.7&&89.8\\ 
		ExistSpace	&	95.3	&	89.7	&	93.2	&&		&	98.9	&71.1	&&92.8\\ 
		GetColor	&	100.0		&	95.8&	99.9	&& &	100.0&	71.4&&	97.9\\ 
		GetColorSpace	&	98.0		&	90.0	&	95.0&	& &	98.2	&71.8&&	92.3\\ 
		GetShape	&	100.0		&	97.3&	99.9&	&	&	100.0  &83.5&&	97.1
		\\ 
		GetShapeSpace	&	97.5	&	89.4	&	93.9	&&&	98.1  &78.7	&&	90.3\\ 
		SimpleCompareShape	&	99.9		&	91.4	&	99.7	&	&&	100.0 & 67.7&&	99.3\\ 
		SimpleCompareColor	&	100.0 		&	91.6  &	99.80&	&	&	100.0&	64.2&&	99.3	  \\ 
		
		
		
		
		
		
		
		\bottomrule
	\end{tabular}
}

	\label{results}
\end{table}



\begin{table}[!t]
	\centering
	\caption{COG Dataset parameters for the canonical setting and the hard setting  }
	\resizebox{\textwidth}{!}{
		
		
		\begin{tabular}{ccccccc}
			\toprule
			
			Dataset    &  	number of frames  &  	maximum memory duration & number of distractors & size of training set & size of validation/test set    \\ 
			\midrule
			
			Canonical setting & 4 & 3 & 1 & 10000320 & 500016 &   \\
			\midrule
			
			Hard  setting & 8 & 7& 10 & 10000320 & 500016  \\
			\bottomrule
			
		\end{tabular}
	}
	
	
	\label{tab:parameters}
\end{table}



	
\end{document}

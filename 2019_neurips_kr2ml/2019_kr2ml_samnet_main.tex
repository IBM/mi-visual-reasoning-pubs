\documentclass{article}
\usepackage{neurips_2019}

\usepackage[utf8]{inputenc} % allow utf-8 input
\usepackage[T1]{fontenc}    % use 8-bit T1 fonts
\usepackage[usenames, svgnames]{xcolor}
\usepackage{hyperref}       % hyperlinks
\usepackage{bchart}
\usepackage{pgfplots}
\pgfplotsset{width=7 cm,compat=1.5}
\hypersetup{
	colorlinks=true,
	linkcolor={red!80!black},
	%	citecolor={blue!80!black},
	%	urlcolor={magenta}
}

\usepackage{url}            % simple URL typesetting
\usepackage{booktabs}       % professional-quality tables
\usepackage{amsfonts}       % blackboard math symbols
\usepackage{nicefrac}       % compact symbols for 1/2, etc.
\usepackage{microtype}      % microtypography
\usepackage{subcaption}
\usepackage{graphicx}

% Useful packages
\usepackage{amsmath, amssymb, amsfonts, bm}
\usepackage{amsthm}
\usepackage{mathtools}
%\usepackage[usenames, dvipsnames]{color}
\usepackage{cleveref}
\crefname{figure}{Figure}{Figures}
\crefname{table}{Table}{Tables}

\usepackage{enumitem}
\setlist{leftmargin=*} 
%\setlist[enumerate]{labelindent=5pt, label=\alph*)} 

\usepackage{wrapfig}
\usepackage{tabularx}
\usepackage{adjustbox}


% \usepackage{blindtext}

\newtheorem{theorem}{Theorem}
\newtheorem{proposition}[theorem]{Proposition}
\newtheorem{corollary}[theorem]{Corollary}
\newtheorem{lemma}[theorem]{Lemma}
\newtheorem{remark}[theorem]{Remark}

\theoremstyle{remark}
\newtheorem*{notation}{Notation}
%\newtheorem*{remark}{Remark}
\newtheorem*{note}{Note}

\theoremstyle{definition}
\newtheorem{definition}[theorem]{Definition}
\newtheorem{fact}{Fact}
\newtheorem*{observation}{Observation}
\newtheorem*{condition}{Condition}
\newtheorem*{claim}{Claim}
\newtheorem*{example}{Example}
\newtheorem*{question}{Question}


\newcommand{\colorbx}[1]{\medskip\noindent\scalebox{1.05}{\fcolorbox{SaddleBrown}{white}{\color{SteelBlue}{\textbf{#1}}}}}

\newcommand{\tk}[1]{\textcolor{red}{TK: #1}}
\newcommand{\ao}[1]{\textcolor{green}{AO: #1}}
\newcommand{\tsj}[1]{\textcolor{magenta}{TSJ: #1}}
\newcommand{\va}[1]{\textcolor{blue}{Vincent A: #1}}

\newcommand{\Reals}{\mathbb{R}}
\newcommand{\T}{\mathsf{T}}
\DeclareMathOperator{\softmax}{\mathrm{softmax}}

\newcommand{\cA}{\mathcal{A}}
\newcommand{\cG}{\mathcal{G}}
\newcommand{\cK}{\mathcal{K}}
\newcommand{\cH}{\mathcal{H}}
\newcommand{\cI}{\mathcal{I}}

\newcommand{\vect}[1]{\mathbf{#1}}
\newcommand{\vx}{\vect{x}}
\newcommand{\vy}{\vect{y}}
\newcommand{\vone}{\vect{1}}
\newcommand{\proj}{\mathrm{proj}}

\newcommand{\imatch}{g^{\mathrm{v}}}
\newcommand{\mmatch}{g^{\mathrm{m}}}

\newcommand{\doadd}{h^{\mathrm{a}}}
\newcommand{\doreplace}{h^{\mathrm{r}}}

\newcommand{\tlast}{\tau^{\mathrm{last}}}
\newcommand{\tlatest}{\tau^{\mathrm{latest}}}
\newcommand{\tnow}{\tau^{\mathrm{now}}}
\newcommand{\tnone}{\tau^{\mathrm{none}}}

\newcommand{\rhead}{rh}
\newcommand{\whead}{wh}


\title{Learning Multi-Step Spatio-Temporal Reasoning with Selective Attention Memory Network}

\author{T.S. Jayram, Tomasz Kornuta, Vincent Albouy, Emre Sevgen, Ahmet Ozcan}

\begin{document}
\maketitle
\begin{abstract}
Visual reasoning in videos requires understanding temporal concepts in addition to the objects and their relations in a given frame.  
%Selective attention and memory are the essential faculties, which humans rely on to accomplish this task.  
%In analogy with human reasoning, we present Selective Attention Memory Network (SAMNet), an end-to-end differentiable recurrent 
%model equipped with external memory.  
To that end,
we present Selective Attention Memory Network (SAMNet), an end-to-end differentiable recurrent 
model equipped with external memory.  
In analogy with human reasoning, 
SAMNet can perform multi-step reasoning on a frame-by-frame basis, and dynamically control information flow to the memory 
to store context-relevant representations and from the memory to answer questions. 
We tested our model on the COG dataset (a multi-frame visual question answering test).
SAMNet outperforms the the original COG model, 
especially on the hardest version of the dataset with longer sequences and a maximum number of distractors.
We also demonstrate that our model has extraordinary generalization capabilities going from easy to hard tasks, without and with additional fine-tuning.
%, and outperformed the state of the art baseline for hard tasks and in terms of generalization over video length and scene complexity.
%	We introduce the Selective Attention Memory Network (SAMNet), a end-to-end differentiable architecture for video reasoning. It is a recurrent model with an external memory that enables frame by frame reasoning over text and video. 
%		We show SAMNet's abilities on the COG dataset made for Video Question Answering (Guangyu Robert Yang, Igor Ganichev et al., ECCV 2018). 
\end{abstract}

\section{Introduction}

In recent years there has been substantial progress in sys-tems  that  can  find  factual  answers  in  text,  starting  withIBM’s Watson system~\cite{ferrucci2010building}

, and now with high-performing neural systems that can answer short ques-tions provided they are given a text that contains the answer e.g.~\cite{wang2018glue}


AI  has  achieved  remarkable  mastery  over  games  such  asChess, Go, and Poker, and evenJeopardy!, but the rich variety of standardized exams has remained a landmark chal-lenge.   Even  in  2016,  the  best  AI  system  achieved  merely 59.3\% on an 8th Grade science exam challenge (Schoenicket al., 2016).




Playing Atari Games~\cite{mnih2015human}


Despite several successes across many domains Deep learning~\cite{lecun2015deep} still struggles with

learning algoritms
learning reasoning~\cite{graves2016hybrid}


Wingrad Scheme challenge~\cite{levesque2012winograd}

visual reasoning~\cite{mogadala2019trends} - datasets such as COG~\cite{yang2018dataset} and 
SVQA (Synthetic Video Question Answering)~\cite{song2018explore}


ARISTO project~\cite{clark2019f} - based on RoBERTa~\cite{liu2019roberta} contextual word embeddings

transformer-based solutions~\cite{vaswani2017attention} using self-attention


“The current neural network approaches will find it difficult to determine which combinations of ‘later’, ‘earlier’, ‘more’, and ‘less’ constitute ‘increase’ and which constitute ‘decrease,'” Davis says. “Neural networks have no inherent idea of magnitude or of time.”
~\cite{davis2016write}

bAbI~\cite{weston2015towards}

Visual Dialog~\cite{das2017visual} - the same, they keep the whole history of the dialog in memory

\begin{itemize}
\item bAbI:  MemNets~\cite{weston2014memory} have access to the whole story at once
\item the same goes to SoftPats~\cite{haurilet2019s} - they build graph per frame and then frame number is treated as one dimensions, so at the end the \textit{Traveler} can access all of them at the same time 
\item The paper~\cite{le2019learning} focuses on SVQA and TGIF-QA -  they access all frames at once, i.e. cut the video into clips, process each frame with CNNs and then aggregate feature representations of equal-size clips obtained by a temporal attention mechanism. So in fact the model has access to all frames all the time.
\end{itemize}
so the time aspect is really... not dealt with?

Additionally, in~\cite{song2018explore} the authors introduced a large-scale dataset caled SVQA (Synthetic Video Question Answering) consisting of (Total QA pairs: 83160/11760/23760 and Total Videos 8400/1200/2400).
As "using all frames is time-consuming. Thus we divide each video into clips (segments) of 16 frames, with 80\% overlap between successive clips (segments)" and "We extract feature from each clip and aggregate features of all clips from one video to form a sequential video representation." -- which means that they identified the problem that you "cannot extract features from all frames" at the beginning and pass that to the model. But instead of proposing a solution that will deal with the video on per-frame basis, they "cheated". ;)

IMPORTANT: \textbf{we do not have any explicit assumptions when it comes to number of frames/length of the movie/number of distractionts}, so there is no need for cutting video into cuts etc.

\subsection{Contributions}

\begin{itemize}
\item \textbf{Time aspect}:
\begin{itemize}
\item Learning the temporal association - grounding the time-related words with meaning
\item Learning the concept of time
\item time context being by-product of gates
\end{itemize}

\item Visual Grounding:
\begin{itemize}
\item Learning complex, multi-step reasoning that involves grounding of words and visual representations/objects
\item 
\end{itemize}
\item \textbf{Selective Attention Memory}:
\begin{itemize}
\item updating the memory content only with relevant visual information depending on the temporal context
\item content based and location based addressing for reading and writing
\item new memory interface/gating designed in such a way enabling the model to control the flow of current visual information and content of the memory in a selective way

\end{itemize}
\end{itemize}









\section{Selective Attention Memory Network (SAMNet)}

%\begin{figure}[!b]
%\begin{minipage}{0.43\textwidth}
%	\centering
%	\includegraphics[width=\textwidth]{../img/architecture/samnet_architecture4}
%\end{minipage}\hfill
%\begin{minipage}{0.55\textwidth}
%	\centering
%	\includegraphics[width=\textwidth]{../img/architecture/samcell_reasoning}
%\end{minipage}\hfill
%\caption{General architecture of SAMNet (left) and a single reasoning step in SAMCell (right)}
%\label{fig:samnet}
%\end{figure}	

%Selective Attention Memory Network (SAMNet) is a end-to-end differentiable model made for video reasoning. It is a model based on attention mechanisms but also on a Selective Attention Memory which is able to store selected entities. This memory enables SAMNet to reason across multiple frames and perform spatio-temporal reasoning. 
%The core of SAMNet is based a recurrent cell called SAMCell. By aligning together a series of k SAMCells per frame, the network can perform k reasoning steps over a frame. At every new frame, a new series of k SAMCells is initiated. The SAMCell can read and write to memory at every frame using a content addressable mechanism. This section describes the model and the different units that composed a SAMCell. They are called the Question-driven Controller, the Visual Retrieval Unit, the Memory Retrieval Unit, the Reasoning unit, the Memory update unit, and the Summary Object Udpate Unit. 
%The model is also composed of an Image Encoder and a Question Encoder both responsible to pre-process the visual and textual inputs. The output unit is a classifier.
%All those modules are described below.

\begin{figure}[hbtp]
	\centering
	\includegraphics[width=\textwidth]{../img/architecture/samnet_architecture4}
	\caption{General architecture of SAMNet}
	\label{fig:samnet}
\end{figure}	

SAMNet is an end-to-end differentiable recurrent model equipped with an external memory for enabling multi-step reasoning over text and video (\cref{fig:samnet}).
The memory is used to store relevant objects representing contextual information about words in text and visual objects in video frames. 
Each address of the memory stores a $d$-dimensional vector, where $d$ is a global parameter.
The memory  can be accessed through either content-addressing, via dot-product attention, or location-based addressing. 
Coupled with gating mechanisms to be described later, this enables correct objects to be retrieved 
in order to perform spatio-temporal reasoning on frames and text. 
%A notable feature of this design is that the number of addresses $N$ can be set to different values during training and 
%testing to fit the characteristics of data.

\begin{figure}[hbtp]
	\centering
	\includegraphics[width=\textwidth]{../img/architecture/samcell_reasoning}
	\caption{Single reasoning step in SAMCell}
	\label{fig:samcell}
\end{figure}	

The core of SAMNet is a recurrent cell called SAMCell (\cref{fig:samcell}). 
By aligning together a new series of $k$ SAMCells per frame, the network can perform $k$ 
reasoning steps over each frame, with information flowing between frames through the external memory. 
While processing a single frame, for $t=1,2, \dots, k$, SAMCell maintains the following information as part of its recurrent state:
(a) $c_t \in \Reals^d$, the control state used to drive the reasoning over objects in the frame and memory; and
(b) $so_t  \in \Reals^d$, the summary visual object representing the relevant object for step $t$.
Let $M_t \in  \Reals^{N \times d}$ denote the external memory with $N$ slots at the end of step $t$.
Let $\whead_t \in  \Reals^N$ denote an attention vector over the memory locations;
in a trained model, $\whead_t$ points to the location of first empty slot in memory for adding new objects.   


This section describes the model and the different units that composed a SAMCell. They are called the Question-driven Controller, the Visual Retrieval Unit, the Memory Retrieval Unit, the Reasoning unit, the Memory update unit, and the Summary Object Udpate Unit. 
The model is also composed of an Image Encoder and a Question Encoder both responsible to pre-process the visual and textual inputs. The output unit is a classifier.
All those modules are described below.














The  Summary Unit is the last unit of the SAMCell. It is responsible to output the new summary object. It first picks which object is relevant between the object extracted from memory and the visual object extracted from the image. Once the relevant object is picked, it is combined with the former summary object through a linear layer to become the new summary object. It is the final step of the SAMCell reasoning cycle. 

The image encoder, question encoder and output unit are described in the appendix.


\section{Experiments}

\subsection{Generalization capabilities}

\section{Summary}

%Observing attention maps shows that SAMNet can effectively perform multi-step reasoning over questions and frames as intended. Despite being trained only on image-question pairs with complex, compositional questions, SAMNet clearly learns to associate visual symbols with words and accurately classify temporal contexts as designed. Besides, the model’s reasoning using neural representations appears to be similar to how a human would operate on abstract symbols when solving the same task, including memorizing and recalling symbols (object embeddings) from the memory when needed. This is not perfect however and the system can sometimes store spurious objects despite the gating and reasoning mechanisms, but still give correct answers. This indicates at least two directions for possible further improvements. The first is to ameliorate content-based addressing with masking, similar to the improvements made for DNC proposed by [2]. Second is to implement variable number of reasoning steps, instead of hard-coded 8 steps, which could utilize Adaptive Computation Time (ACT) [5].

Even though transfer learning in computer vision is a common practice, a holistic view of its impact on visual reasoning was missing.
To capture and quantify the influence of transfer learning on visual reasoning, we proposed a new taxonomy, articulated around three aspects: feature, temporal and reasoning transfer.  This enabled us to reuse the existing datasets for image and video QA, namely CLEVR and COG, and isolate splits better capturing reasoning transfer.
We note that some of the proposed splits form tasks (e.g. Train on all but \textit{t}) which are complementary to well established ones in the literature, e.g. in Taskonomy~\cite{zamir2018taskonomy}.

Our experiments on transfer learning showed the shortcomings of existing approaches, especially for video reasoning.  Hence, we designed a novel Memory-Augmented Neural Network model called SAMNet, with mechanisms to address these deficiencies.
SAMNet showed significant improvements over SOTA models on the COG dataset for Video Reasoning  and achieved comparable performance on the CLEVR dataset for Image Reasoning.
It also demonstrated excellent generalization capabilities for temporal and reasoning transfer. Moreover, through the cautious use of fine-tuning, SAMNet's performance advanced even further.
We hope that the proposed taxonomy, newly established reasoning transfer tasks, along with the provided baselines will bolster new research on both models and datasets for transfer learning on visual reasoning.


\section*{Acknowledgement}
The authors would like to thank to the authors of COG paper (Igor Ganichev in particular) for sharing the detailed results with performances achieved by their COG baseline model.
	
%\newpage
\bibliographystyle{abbrv}
\bibliography{../cog_bibliography}

\newpage
\appendix

 
 \section{Full MAC and S-MAC comparison}
\label{sec:full_comparison}

In \tableref{tab:results_full} we present the full comparison between MAC and S-MAC models achieved with our implementations of both models.
In the [Row] column we indicate the measures that we have  analyzed and discussed in the experiments section of the main paper.


\begin{table}[!h]
	\centering
	\begin{tabular}{ccccCcCcc}
		\toprule
		\multirow{2}{*}{Model} & \multicolumn{3}{c}{Training} &  \multicolumn{2}{c}{Fine-tuning} & \multicolumn{2}{c}{Test} & \multirow{2}{*}{Row} \\
		\cmidrule{2-4} \cmidrule{5-6} \cmidrule{7-8}
		& Dataset                & Time [h:m] & Acc [\%]          & Dataset & Acc [\%]  & Dataset & Acc [\%] & \\
		\midrule
		\multirow{15}{*}{MAC} & \multirow{10}{*}{CLEVR}  & \multirow{10}{*}{30:52}  & \multirow{10}{*}{96.70} & \multirow{4}{*}{--}   & \multirow{4}{*}{--}  & CLEVR    & 96.17    & (a)      \\
		\cmidrule{7-8} 
		&                        &  &               &     &                                & CoGenT-A    &  96.22 &  \\
		\cmidrule{7-8} 
		&                        &   &              &     &                               & CoGenT-B   & 96.27 & \\
		
		\cmidrule{5-6} \cmidrule{7-8} 
		&                             &                                         &    &   \multirow{2}{*}{CoGenT-A}         &       \multirow{2}{*}{98.06}          & CoGenT-A &  94.60	   &      \\
		\cmidrule{7-8} 
		&                             &                                         &       &         &                & CoGenT-B &    93.28   &    \\
		\cmidrule{5-6} \cmidrule{7-8} 
		&                             &                                         &    &   \multirow{2}{*}{CoGenT-B}         &       \multirow{2}{*}{98.16}          & CoGenT-A &  93.02    &     \\
		\cmidrule{7-8} 
		&                             &                                         &       &         &                & CoGenT-B &    94.44   &    \\  
		
		\cmidrule{2-4} \cmidrule{5-6} \cmidrule{7-8} 
		& \multirow{5}{*}{CoGenT-A} & \multirow{5}{*}{30:52}     & \multirow{5}{*}{97.02}   &  \multirow{2}{*}{--}  &  \multirow{2}{*}{--}    & CoGenT-A & 96.88    &     \\
		\cmidrule{7-8} 
		&                             &                                         &       &         &                & CoGenT-B & 79.54     &    \\
		\cmidrule{5-6} \cmidrule{7-8} 
		&                             &                                         &    &   \multirow{2}{*}{CoGenT-B}         &       \multirow{2}{*}{97.91}          & CoGenT-A &  92.06     &    \\
		\cmidrule{7-8} 
		&                             &                                         &       &         &                & CoGenT-B &    95.62   &    \\
		\midrule
		\multirow{15}{*}{S-MAC} & \multirow{10}{*}{CLEVR}  & \multirow{10}{*}{28:30}  & \multirow{10}{*}{95.82} & \multirow{3}{*}{--}   & \multirow{3}{*}{--}  & CLEVR    & 95.29     & (b)      \\
		\cmidrule{7-8} 
		&                        &  &               &     &                                & CoGenT-A    &  95.47 & (d)  \\
		\cmidrule{7-8} 
		&                        &   &              &     &                               & CoGenT-B   &  95.58 & (e) \\		
		
		\cmidrule{5-6} \cmidrule{7-8} 
		&                             &                                         &    &   \multirow{2}{*}{CoGenT-A}         &       \multirow{2}{*}{97.48}          & CoGenT-A &  93.44      &   \\
		\cmidrule{7-8} 
		&                             &                                         &       &         &                & CoGenT-B &    92.31   &    \\
		\cmidrule{5-6} \cmidrule{7-8} 
		&                             &                                         &    &   \multirow{2}{*}{CoGenT-B}         &       \multirow{2}{*}{97.67}          & CoGenT-A &  92.11     & (i)    \\
		\cmidrule{7-8} 
		&                             &                                         &       &         &                & CoGenT-B &    92.95  & (j)     \\  		
		
		\cmidrule{2-4} \cmidrule{5-6} \cmidrule{7-8} 
		& \multirow{5}{*}{CoGenT-A}   & \multirow{5}{*}{28:33}   & \multirow{5}{*}{96.09}  &  \multirow{2}{*}{--}  &  \multirow{2}{*}{--}   & CoGenT-A & 95.91    & (c)      \\
		\cmidrule{7-8} 
		&                             &                                         &     &          &                & CogenT-B & 78.71       & (f)   \\
		\cmidrule{5-6} \cmidrule{7-8} 
		&                             &                                         &    &   \multirow{2}{*}{CoGenT-B}         &       \multirow{2}{*}{96.85}          & CoGenT-A &  91.24   & (g)      \\
		\cmidrule{7-8} 
		&                             &                                         &       &         &                & CoGenT-B &    94.55   & (h)    \\
		\bottomrule
	\end{tabular}
	\caption{CLEVR \& CoGenT accuracies for the MAC \& S-MAC models.}
	\label{tab:results_full}
\end{table}



\end{document}

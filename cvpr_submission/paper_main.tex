\documentclass[10pt,twocolumn,letterpaper]{article}

\usepackage{cvpr}
\usepackage{times}
\usepackage{epsfig}
\usepackage{graphicx}
\usepackage[font=small]{subcaption}
\captionsetup[figure]{skip=5pt}
\captionsetup[table]{skip=5pt}
\setlength{\belowcaptionskip}{-15pt}  % reduce padding below figure
\usepackage{amsmath, amssymb, amsfonts,bm}
\usepackage{amsthm}

% Include other packages here, before hyperref.
\usepackage{booktabs}       % professional-quality tables
\usepackage{multirow}
\usepackage{adjustbox}
\usepackage{lipsum} % for dummy text only

\usepackage{url}            % simple URL typesetting
% If you comment hyperref and then uncomment it, you should delete
% egpaper.aux before re-running latex.  (Or just hit 'q' on the first latex
% run, let it finish, and you should be clear).
\usepackage[pagebackref=true,breaklinks=true,colorlinks,bookmarks=false]{hyperref}

%\cvprfinalcopy % *** Uncomment this line for the final submission

\def\cvprPaperID{7813} % *** Enter the CVPR Paper ID here
\def\httilde{\mbox{\tt\raisebox{-.5ex}{\symbol{126}}}}

% Pages are numbered in submission mode, and unnumbered in camera-ready
\ifcvprfinal\pagestyle{empty}\fi

%\usepackage{booktabs}       % professional-quality tables
%\usepackage{amsfonts}       % blackboard math symbols
%\usepackage{nicefrac}       % compact symbols for 1/2, etc.
%\usepackage{microtype}      % microtypography
%\usepackage{subcaption}
%\usepackage{graphicx}

% Useful packages
%\usepackage{mathtools}
%\usepackage[usenames, dvipsnames]{color}

\usepackage{cleveref}
\crefname{section}{Section}{Sections}
\crefname{figure}{Figure}{Figures}
\crefname{table}{Table}{Tables}
\crefname{appendix}{Appendix}{Appendices}
\crefname{definition}{Definition}{Definitions}

\newtheorem{theorem}{Theorem}
\newtheorem{proposition}[theorem]{Proposition}
\newtheorem{corollary}[theorem]{Corollary}
\newtheorem{lemma}[theorem]{Lemma}
\newtheorem{remark}[theorem]{Remark}
\newtheorem{definition}[theorem]{Definition}

\theoremstyle{remark}
\newtheorem*{notation}{Notation}
%\newtheorem*{remark}{Remark}
\newtheorem*{note}{Note}

\newtheorem{fact}{Fact}
\newtheorem*{observation}{Observation}
\newtheorem*{condition}{Condition}
\newtheorem*{claim}{Claim}
\newtheorem*{example}{Example}
\newtheorem*{question}{Question}


\usepackage{enumitem}
\setlist{leftmargin=*} 
%\setlist[enumerate]{labelindent=5pt, label=\alph*)} 

\usepackage{wrapfig}
\usepackage{tabularx}
\usepackage{adjustbox}



% \usepackage{blindtext}



%% User's macros %%
% \newcommand{\colorbx}[1]{\medskip\noindent\scalebox{1.05}{\fcolorbox{SaddleBrown}{white}{\color{SteelBlue}{\textbf{#1}}}}}

\newcommand{\Reals}{\mathbb{R}}
\newcommand{\cD}{\mathcal{D}}
\newcommand{\cT}{\mathcal{T}}
\newcommand{\cX}{\mathcal{X}}
\newcommand{\cY}{\mathcal{Y}}
\newcommand{\cQ}{\mathcal{Q}}
\newcommand{\cV}{\mathcal{V}}
\newcommand{\cF}{\mathcal{F}}

\renewcommand{\vec}[1]{\mathbf{#1}}
\newcommand{\vc}{\vec{c}}
\newcommand{\vso}{\vec{so}}
\newcommand{\vone}{\vec{1}}
\newcommand{\proj}{\mathrm{proj}}
\newcommand{\vx}{\vec{x}}
\newcommand{\vy}{\vec{y}}

\newcommand{\imatch}{g^{\mathrm{v}}}
\newcommand{\mmatch}{g^{\mathrm{m}}}

\newcommand{\doadd}{h^{\mathrm{a}}}
\newcommand{\doreplace}{h^{\mathrm{r}}}

\newcommand{\tlast}{\tau^{\mathrm{last}}}
\newcommand{\tlatest}{\tau^{\mathrm{latest}}}
\newcommand{\tnow}{\tau^{\mathrm{now}}}
\newcommand{\tnone}{\tau^{\mathrm{none}}}

\newcommand{\rhead}{\vec{rh}}
\newcommand{\whead}{\vec{wh}}


\newcommand{\tk}[1]{\textcolor{red}{TK: #1}}
\newcommand{\ao}[1]{\textcolor{green}{AO: #1}}
\newcommand{\tsj}[1]{\textcolor{magenta}{TSJ: #1}}
\newcommand{\vm}[1]{\textcolor{blue}{VM: #1}}

\newcommand{\compresslist}{ % Define a command to reduce spacing within itemize/enumerate environments, this is used right after \begin{enumerate} or \begin{enumerate}
\setlength{\itemsep}{1pt}
\setlength{\parskip}{0pt}
\setlength{\parsep}{0pt}}



%%%%%%%%%%%%%%%

\begin{document}

%%%%%%%%% TITLE
\title{Transfer Learning in Multimodal Compositional and Relational Reasoning}

%%%%%%%%% AUTHORS
% WILL NEED TO REARRANGE THE AUTHORS LIST & GET CORRECT INSTITUTIONS / EMAILS
\author{T.S. Jayram \and Vincent Albouy \and Tomasz Kornuta  \and Emre Sevgen \and Ahmet Ozcan \and Vincent Marois\\
IBM Research AI\\
Almaden Research Center, San Jose, CA 95120, USA\\
Thomas J. Watson Research Center, Yorktown Heights, NY 10598, USA\\
{\tt\small {jayram,  asozcan}@us.ibm.com}
{\tt\small {tkornuta, sesevgen}@gmail.com}
{\tt\small vincent.marois@ibm.com}
}

\maketitle

%%%%%%%%% ABSTRACT
\begin{abstract}
  % \lipsum[1]
  Transfer learning is becoming the de facto solution for vision and text encoders in the front-end processing of machine learning solutions.  Utilizing vast amounts of knowledge in pre-trained models and subsequent fine-tuning allows achieving higher accuracy in domains where labeled data are limited.  In this paper, we analyzed the efficiency of transfer learning in multimodal (vision and language) reasoning by introducing a new model (SAMNet) and testing it on two datasets: COG and CLEVR.  Our new model achieved state-of-the-art accuracy and showed significantly better generalization capabilities compared to the baseline.    Based on extensive experimentation on transfer learning on these datasets, we also formalized a taxonomy of transfer learning around three areas: feature, temporal, and reasoning transfer.   
\end{abstract}

%%%%%%%%% BODY TEXT
\section{Introduction}

Integration of vision and language in deep neural network models allows the system to learn joint representations of objects, concepts, and relations.  Potentially, this approach can lead us towards Harnad's \textit{symbol grounding problem}~\cite{harnad2003symbol} but we are quite far from achieving the full capabilities of visually grounded language learning.

Recently, there is a growing interest in neuro-symbolic approaches, which can combine the power of representation learning and symbolic logic that is interpretable \cite{mao2019neurosymbolic}. These approaches focus on \textit{symbol manipulation} rather than learning \textit{grounded symbols}.  Furthermore, symbolic priors (e.g., domain knowledge) and integration of logic depend on hand-crafted modules.  In the near term, this direction is certainly promising and can address some of the shortcomings of machine learning (i.e., the lack of explainability)\cite{vedantam2019probabilistic}.  However, in the long run, the desire is to learn grounded representations, which may lead to the emergence of symbols \cite{taniguchi2018symbol}.

Starting with Image Question Answering~\cite{malinowski2014multi,antol2015} and Image Captioning~\cite{karpathy2015deep}, a variety of tasks that integrate vision and language have appeared in the past several years~\cite{mogadala2019trends}. 
Those directions include e.g., Video QA~\cite{MovieQA} and Video Action Recognition~\cite{monfort2019moments}, that provide an additional challenge of understanding \emph{temporal} aspects, and Video Reasoning~\cite{song2018explore,yang2018dataset}, that tackles both spatial (comparison of object attributes, counting and other relational question) and temporal aspects and relations (e.g. object disappearance).
To deal with the temporal aspect most studies typically cut the whole video into clips; e.g., in~\cite{song2018explore} the model extracts visual features from each frame and aggregates features first into clips, followed by aggregation over clips to form a single video representation.
Still, when reasoning and producing the answer, the system in fact has \textit{access to all frames}. 
Similarly, in Visual Dialog~\cite{das2017visual} the system memorizes the whole dialog history.
However, in real-time dialog or video monitoring, it is not always possible to keep the entire history of conversation nor all frames from the beginning of the recording.  


As evident from human cognition, attention and memory are the key competencies required to solve these problems, and unsurprisingly, the AI research is rapidly growing in these areas.
The ability to deal with temporal casuality can pose a challenge for also in pure natural language processing (NLP) settings, e.g. in question answering (QA) and dialog applications.  
Current NLP solutions, in many problem settings, work around this challenge by processing the entire text input and reason over it multiple times using attention \cite{vaswani2017attention} or other mechanisms.
For example, typical solutions to the bAbI reasoning task, such as Memory Networks \cite{weston2014memory}, involve processing all the supporting facts at once and keeping them in memory all the time while searching for the answer.


\paragraph{Contributions}


\section{Related work}
\label{sec:related_work}

In Computer Vision, it is now standard practice to pretrain an image encoder (such as VGG~\cite{simonyan2014very} or ResNet~\cite{he2016deep}) on large-scale datasets (such as ImageNet~\cite{deng2009imagenet}), and reuse the weights in unrelated domains and tasks, such as segmentation of cars~\cite{iglovikov2018ternausnet} or Visual Question Answering (VQA) in a medical domain~\cite{kornuta2019leveraging}.
Such performance improvements are appealing, especially in cases where both the domain (natural vs. medical images) and the task (image classification vs. image segmentation vs VQA) change significantly.

Similar developments have emerged in the Natural Language Processing (NLP) community.
Using shallow word embeddings, such as word2vec~\cite{mikolov2013distributed} or GloVe~\cite{pennington2014glove}, pretrained on large corpuses from e.g.\ Wikipedia or Twitter, was de facto a standard procedure accross different NLP domains and tasks for the last several years.
Recently, there is a clear, growing trend of utilization of deep contextualized word representations such as ELMo~\cite{peters2018deep} (based on bidirectional LSTMs~\cite{hochreiter1997long}) or BERT~\cite{devlin2018bert} (based on the Transformer~\cite{vaswani2017attention} architecture), where entire deep networks (not just the input layer) are pretrained on very large corporas.
%In analogy to repositories with pretrained image encoders in TensorFlow~\cite{} or PyTorch~\cite{}, the NLP community has also started to create model repositories, some with dozens of pretrained models ready to be downloaded and used. HuggingFace~\cite{wolf2019transformers} is one of the most notable examples.

Visual Reasoning tasks combining both the visual and language modalities~\cite{mogadala2019trends} naturally draw from those findings by reusing the pretrained image and word/question encoders.
The community also seems to be realising that the models should not only be able to generalize on the low-level image features, but also on the performed high-level, abstract \textit{reasoning}.
First notable attemp to establish a framework enabling transfer learning of reasoning skill was CLEVR~\cite{johnson2017clevr} with its two CoGenT variants.
As a result several interesting works appeared~\cite{mascharka2018transparency, perez2018film, johnson2017inferring,marois2018transfer}, proposing models trying to cope with transfer from one to the other domain with different distributions of visual attribute combinations.
Similarly, the baseline model introduced along with the COG dataset~\cite{yang2018dataset} have shown that it can do significantly better answer questions belonging to task classes not being explicitly trained on when leveraging knowledge learned on other tasks.
These results clearly indicate the usefulness of transfer learning, despite the assumption of similar distributions between the source and target domains not being respected.
At the same time transfer learning several research questions, such as the characteristics which make a whole dataset or a particular task more favorable to be used in pretraining (notably ImageNet~\cite{huh2016makes}), or regarding the observed performance correlation of models with different architectures between the source and target domains~\cite{kornblith2019better}.
One of the most systematic works in this area is the computational taxonomic map for task transfer learning~\cite{zamir2018taskonomy}, aiming at discovering the dependencies between twenty-six different computer vision tasks.
In this work we extend that idea further and introduce a taxonomy enabling us to isolate and quantify different aspects of visual reasoning on the performance of our model.
%We illustrate it with the two above mentioned CLEVR and COG datasets.
%The taxonomy enabled us to perform more systematic studies and facilitate comparison between the already existing solutions and the newly introduced SAMNet model.


%transfer learning has helped models generalize better to domains with different distributions. 


%We propose to address this by introducing a taxonomy of transfer learning in visual reasoning, and illustrate it with two visual reasoning datasets.


% \tk{The experimental results clearly indicate that transfer learning is helping, despite that the core assumptions on the compatibility/similarity of domains is broken (transfering between domains having similar distribution).
% We do not understand this fully, thus the need for a more systematic research on that topic emerges.
% The paper adresses that by introducing a theoretical framework/taxonomy enabling to categorise visual reasoning tasks.
% As a starting point, we picked two visual reasoning datasets and analyse the achieved results through the prism of the proposed taxonomy.
% }
%
%
% \tk{after reading that section the reader should end up with a conclusion that: there are no good models for TL in VR and, moreover, the datasets are randomly testing this or that, there is no theoretical framework showing that do they mean/bigger picture is missing}


\section{Transfer Learning}

\begin{figure*}
	\centering
	\includegraphics[width=0.9\textwidth]{../img/architecture/transfer_taxo}
	\caption{Transfer learning taxonomy.}\label{fig:taskonomy}
\end{figure*}
Due to the complex nature of visual reasoning, it turns out that
transfer learning from a source domain to a target domain can be investigated in 
multiple ways depend on how the features are related between the two domains, including attributes such as size, color and shape, as well as temporal characteristics reflected in the number of frames and the scene complexity.
To establish a formal framework for these various aspects, we first recall the basic definition of transfer learning~\cite{pan2009survey}.

A \emph{domain} is a pair $\cD = (\cX,P(X))$, where $\cX$ is a feature space and $P(X)$ is a marginal probability distribution.
For visual reasoning problems considered in this paper, 
$\cX$ will consist of purely visual inputs, i.e., either images or videos in some cases, or 
a combination of both visual inputs and questions in other cases. 
A \emph{task} is a pair $\cT= (\cY,f(\cdot))$, where $\cY$ is a label space and $f: \cX \to \cY$ is a prediction function. 
When the domain elements consist of both the question and the visual input, there is only one task, namely, to answer the 
question\footnote{%
	For the COG dataset, the answer is a tuple, one for each frame in the video, whereas for typical video answering datasets,
	only a single answer is needed for the entire video.}. % 
If the domain elements consist of just the visual inputs, then the task is defined by the question so that each question 
defines a separate task.

\begin{definition}[\cite{pan2009survey}]
	\label{defn:transfer}
	Given a source domain $\cD_S$ and a source learning task $\cT_S$, a target domain $\cD_T$ and a target learning task $\cT_T$, transfer learning aims to help improve the 
	learning of the target predictive function $f_T(\cdot)$ in $\cD_T$ using the knowledge  in $\cD_S$ and $\cT_S$, where $\cD_S \ne \cD_T$, or $\cT_S \ne \cT_T$.
\end{definition}
In all our applications, $\cX_S = \cX_T$, so $\cD_S \ne \cD_T$ means that the marginal distributions $P_S$ and $P_T$ are different.
Similarly, $\cT_S \ne \cT_T$ means that either $Y_S \ne Y_T$ or that the associated prediction functions are different.

Although~\cref{defn:transfer} is quite general, it does not adequately capture all artifacts present in visual reasoning.
For example, consider the transfer learning setting where the tasks $\cT_S$ and $\cT_T$ are the same 
but the marginal distributions $P_S$ and $P_T$ are different (referred to as \emph{domain adaptation}).
As mentioned in the introduction, one setting is the case of static images, 
where this could be due to having different feature combinations in the source and target.
A different setting is in the context of video reasoning where the number of frames can increase significantly going from source to target.
These require possibly very different methods: the first involves building disentangled feature representations that can generalize across 
domains; the second might need external memory to remember relevant objects to generalize across frame lengths.
Another situation is when the questions themselves can be grouped into families such as count-based queries, 
comparison of objects, or existence of objects with certain features etc.
This entails studying transfer learning between families of tasks which requires extending the above definition.

We now formally define 3 kinds of transfer learning problems, namely,
\emph{feature transfer}, \emph{temporal transfer}, 
and \emph{reasoning transfer}. 
These are illustrated in~\cref{fig:taskonomy} using representative examples from CLEVR-CoGenT and COG datasets that are particularly suited for experimental
investigations of transfer learning.
In the next section we detail the performance of
SAMNet for each of these 3 kinds, using the CLEVR-CoGenT dataset
for feature transfer, the COG dataset for temporal transfer, and finally
both datasets for reasoning transfer, with appropriate baseline comparisons.

Let $\cQ$ and $\cV$ denote the set of questions and visual inputs, respectively.
\begin{description}
	%\compresslist % used to reduce spacing
	\item[Feature Transfer:] In this setting of domain adaptation, $\cX_S = \cX_T \subseteq \cQ \times \cV$
	and the task $f(q,v)$ is just the answer to the question $q$ on visual input $v$. The output set $\cY$ is the union of legitimate answers
	over all questions in $\cQ$.
	The marginal distributions $P_S$ and $P_T$ differ in the feature attributes such as shape, color, and size, or their combinations
	thereof.
	
	\item[Temporal Transfer:] This setting is similar to attribute adaptation in that $\cX_S = \cX_T \subseteq \cQ \times \cV$
	and there is a single task.
	The key difference is that we introduce a notion of complexity $C(v) = (n, m)$ for a visual input $v$,
	where $n$ equals the maximum number of objects $n$ in an image, and $m$
	equals  the number of frames in a video. 
	For any visual input $v_S$ coming from $\cX_S$ with $C(v_S) = (n_S, m_S)$
	and for any visual input $v_T$ coming from $\cX_T$ with $C(v_T) = (n_T, m_T)$, we require that $n_T \ge n_S$ and 
	$m_T \ge m_S$ with at least one inequality being a strict one. 
	Thus, we necessarily increase the complexity of the visual input going from the source to the target domain.
	
	\item[Reasoning Transfer:]
	This setting requires an extension of~\cref{defn:transfer} above to investigate transfer learning when
	grouping questions into families. Let $\cV$ be the feature space consisting of visual inputs only, shared by
	all tasks, with a common marginal distribution $P(X)$. For each question $q \in \cQ$, we define the task 
	$\cT_q = (\cY_q, f_q(\cdot))$ where
	the output set $\cY_q$ is the set of legitimate answers to $q$ and $f_q(v)$, for a visual input $v$, 
	is the answer to question $q$ on visual input $v$.
	Thus, tasks are in a 1-1 correspondence with questions.
	A \emph{task family} is a probability distribution on tasks which in our case can be obtained by defining the distribution on $\cQ$. 
	Given a task family, the goal is to learn a prediction function that gives an answer to $f_q(v)$ for $v \in \cV$ chosen according 
	to the feature space distribution and $q$ chosen according to the task probability distribution.
	Suppose $\cF_S$ is the source task family and $\cF_T$ is the target task family.
	Transfer learning aims to help improve the learning of the predictive function for the target task family 
	using the knowledge in the source task family.
	
\end{description}

If labeled data is available for $\cX_T$, a training algorithm distinction we make is between \emph{zero-shot learning} and \emph{finetuning}. Finetuning entails the use of labeled data in the target domain $\cD_T$, foreseeing performance gain on the target task $\cX_T$, after initial training on $\cX_S$ and additional training on $\cX_T$. Zero-shot learning thus refers to immediate test on $\cX_T$ after initial training on $\cX_S$.

\section{Selective Attention Memory (SAM) Network}

\tk{Description of the SAMNet}

\section{Experiments}

\subsection{Generalization capabilities}

\section{Summary}

%Observing attention maps shows that SAMNet can effectively perform multi-step reasoning over questions and frames as intended. Despite being trained only on image-question pairs with complex, compositional questions, SAMNet clearly learns to associate visual symbols with words and accurately classify temporal contexts as designed. Besides, the model’s reasoning using neural representations appears to be similar to how a human would operate on abstract symbols when solving the same task, including memorizing and recalling symbols (object embeddings) from the memory when needed. This is not perfect however and the system can sometimes store spurious objects despite the gating and reasoning mechanisms, but still give correct answers. This indicates at least two directions for possible further improvements. The first is to ameliorate content-based addressing with masking, similar to the improvements made for DNC proposed by [2]. Second is to implement variable number of reasoning steps, instead of hard-coded 8 steps, which could utilize Adaptive Computation Time (ACT) [5].

Even though transfer learning in computer vision is a common practice, a holistic view of its impact on visual reasoning was missing.
To capture and quantify the influence of transfer learning on visual reasoning, we proposed a new taxonomy, articulated around three aspects: feature, temporal and reasoning transfer.  This enabled us to reuse the existing datasets for image and video QA, namely CLEVR and COG, and isolate splits better capturing reasoning transfer.
We note that some of the proposed splits form tasks (e.g. Train on all but \textit{t}) which are complementary to well established ones in the literature, e.g. in Taskonomy~\cite{zamir2018taskonomy}.

Our experiments on transfer learning showed the shortcomings of existing approaches, especially for video reasoning.  Hence, we designed a novel Memory-Augmented Neural Network model called SAMNet, with mechanisms to address these deficiencies.
SAMNet showed significant improvements over SOTA models on the COG dataset for Video Reasoning  and achieved comparable performance on the CLEVR dataset for Image Reasoning.
It also demonstrated excellent generalization capabilities for temporal and reasoning transfer. Moreover, through the cautious use of fine-tuning, SAMNet's performance advanced even further.
We hope that the proposed taxonomy, newly established reasoning transfer tasks, along with the provided baselines will bolster new research on both models and datasets for transfer learning on visual reasoning.


{\small
\bibliographystyle{ieee_fullname}
\bibliography{bibliography}
}

\clearpage
\appendix

 
 \section{Full MAC and S-MAC comparison}
\label{sec:full_comparison}

In \tableref{tab:results_full} we present the full comparison between MAC and S-MAC models achieved with our implementations of both models.
In the [Row] column we indicate the measures that we have  analyzed and discussed in the experiments section of the main paper.


\begin{table}[!h]
	\centering
	\begin{tabular}{ccccCcCcc}
		\toprule
		\multirow{2}{*}{Model} & \multicolumn{3}{c}{Training} &  \multicolumn{2}{c}{Fine-tuning} & \multicolumn{2}{c}{Test} & \multirow{2}{*}{Row} \\
		\cmidrule{2-4} \cmidrule{5-6} \cmidrule{7-8}
		& Dataset                & Time [h:m] & Acc [\%]          & Dataset & Acc [\%]  & Dataset & Acc [\%] & \\
		\midrule
		\multirow{15}{*}{MAC} & \multirow{10}{*}{CLEVR}  & \multirow{10}{*}{30:52}  & \multirow{10}{*}{96.70} & \multirow{4}{*}{--}   & \multirow{4}{*}{--}  & CLEVR    & 96.17    & (a)      \\
		\cmidrule{7-8} 
		&                        &  &               &     &                                & CoGenT-A    &  96.22 &  \\
		\cmidrule{7-8} 
		&                        &   &              &     &                               & CoGenT-B   & 96.27 & \\
		
		\cmidrule{5-6} \cmidrule{7-8} 
		&                             &                                         &    &   \multirow{2}{*}{CoGenT-A}         &       \multirow{2}{*}{98.06}          & CoGenT-A &  94.60	   &      \\
		\cmidrule{7-8} 
		&                             &                                         &       &         &                & CoGenT-B &    93.28   &    \\
		\cmidrule{5-6} \cmidrule{7-8} 
		&                             &                                         &    &   \multirow{2}{*}{CoGenT-B}         &       \multirow{2}{*}{98.16}          & CoGenT-A &  93.02    &     \\
		\cmidrule{7-8} 
		&                             &                                         &       &         &                & CoGenT-B &    94.44   &    \\  
		
		\cmidrule{2-4} \cmidrule{5-6} \cmidrule{7-8} 
		& \multirow{5}{*}{CoGenT-A} & \multirow{5}{*}{30:52}     & \multirow{5}{*}{97.02}   &  \multirow{2}{*}{--}  &  \multirow{2}{*}{--}    & CoGenT-A & 96.88    &     \\
		\cmidrule{7-8} 
		&                             &                                         &       &         &                & CoGenT-B & 79.54     &    \\
		\cmidrule{5-6} \cmidrule{7-8} 
		&                             &                                         &    &   \multirow{2}{*}{CoGenT-B}         &       \multirow{2}{*}{97.91}          & CoGenT-A &  92.06     &    \\
		\cmidrule{7-8} 
		&                             &                                         &       &         &                & CoGenT-B &    95.62   &    \\
		\midrule
		\multirow{15}{*}{S-MAC} & \multirow{10}{*}{CLEVR}  & \multirow{10}{*}{28:30}  & \multirow{10}{*}{95.82} & \multirow{3}{*}{--}   & \multirow{3}{*}{--}  & CLEVR    & 95.29     & (b)      \\
		\cmidrule{7-8} 
		&                        &  &               &     &                                & CoGenT-A    &  95.47 & (d)  \\
		\cmidrule{7-8} 
		&                        &   &              &     &                               & CoGenT-B   &  95.58 & (e) \\		
		
		\cmidrule{5-6} \cmidrule{7-8} 
		&                             &                                         &    &   \multirow{2}{*}{CoGenT-A}         &       \multirow{2}{*}{97.48}          & CoGenT-A &  93.44      &   \\
		\cmidrule{7-8} 
		&                             &                                         &       &         &                & CoGenT-B &    92.31   &    \\
		\cmidrule{5-6} \cmidrule{7-8} 
		&                             &                                         &    &   \multirow{2}{*}{CoGenT-B}         &       \multirow{2}{*}{97.67}          & CoGenT-A &  92.11     & (i)    \\
		\cmidrule{7-8} 
		&                             &                                         &       &         &                & CoGenT-B &    92.95  & (j)     \\  		
		
		\cmidrule{2-4} \cmidrule{5-6} \cmidrule{7-8} 
		& \multirow{5}{*}{CoGenT-A}   & \multirow{5}{*}{28:33}   & \multirow{5}{*}{96.09}  &  \multirow{2}{*}{--}  &  \multirow{2}{*}{--}   & CoGenT-A & 95.91    & (c)      \\
		\cmidrule{7-8} 
		&                             &                                         &     &          &                & CogenT-B & 78.71       & (f)   \\
		\cmidrule{5-6} \cmidrule{7-8} 
		&                             &                                         &    &   \multirow{2}{*}{CoGenT-B}         &       \multirow{2}{*}{96.85}          & CoGenT-A &  91.24   & (g)      \\
		\cmidrule{7-8} 
		&                             &                                         &       &         &                & CoGenT-B &    94.55   & (h)    \\
		\bottomrule
	\end{tabular}
	\caption{CLEVR \& CoGenT accuracies for the MAC \& S-MAC models.}
	\label{tab:results_full}
\end{table}



\end{document}
